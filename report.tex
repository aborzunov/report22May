\documentclass{beamer}

\usetheme{Madrid}
\usecolortheme{beaver}
\setbeamertemplate{caption}[numbered]
\makeatother
\setbeamertemplate{footline}
{\leavevmode%
  \hbox{%
  \begin{beamercolorbox}[wd=.4\paperwidth,ht=2.25ex,dp=1ex,center]{author in head/foot}%
      \usebeamerfont{author in head/foot}\insertshortauthor{}~(\insertshortinstitute{})
  \end{beamercolorbox}%
  \begin{beamercolorbox}[wd=.5\paperwidth,ht=2.25ex,dp=1ex,center]{title in head/foot}%
    \usebeamerfont{title in head/foot}\insertshorttitle
  \end{beamercolorbox}%
  \begin{beamercolorbox}[wd=.1\paperwidth,ht=2.25ex,dp=1ex,center]{date in head/foot}%
    \insertframenumber{} / \inserttotalframenumber\hspace*{1ex}
  \end{beamercolorbox}}%
}
\makeatletter
\setbeamertemplate{navigation symbols}{}

\usepackage[utf8]{inputenc}
\usepackage[english, russian]{babel}

\usepackage{animate}
\usepackage{graphicx}
\bibliographystyle{unsrt}

\usepackage{hyperref}
\hypersetup{colorlinks=true,
    linkcolor=blue,
    filecolor=magenta,
    urlcolor=cyan,
}

\usepackage{amssymb}
\usepackage{amsmath}
\usepackage{amsthm}
\usepackage{lscape}
\usepackage{color}
\graphicspath{{./images/}}


%Information to be included in the title page:
\title[Восстановление топографии поверхности]{Обратная задача восстановления формы отражающей поверхности методом обратно рассеянных электронов}
\author[А.А. Борзунов]{А.А.~Борзунов \inst{1} \and В.Ю.~Караулов \inst{1} \and Н.А.~Кошев \inst{2} \and Д.В.~Лукьяненко \inst{1} \and Э.И.~Рау \inst{1} \and А.Г.~Ягола \inst{1} \and С.В.~Зайцев \inst{1}}
\institute[МГУ им М.В. Ломоносова]{\inst{1}Московский Государственный Университет им. М.В. Ломоносова \and \inst{2}Сколковский институт науки и технологий }
\titlegraphic{\includegraphics[width=2cm]{ff.png}~\includegraphics[width=2cm]{Skoltech-logo.png}
}

\date{22 мая 2019 г.}



\begin{document}

\begin{frame}
    \titlepage
\end{frame}

\begin{frame}
    \frametitle{Проблематика}
    \begin{columns}
        \begin{column}{0.5\textwidth}
            \includegraphics[width=1.0\linewidth]{VA.png}
        \end{column}
        \begin{column}{0.5\textwidth}
            % nothing here, will be in the next frame
        \end{column}
    \end{columns}
\end{frame}

% Создаем анимацию посредством копирования слайда :-)
\begin{frame}
    \frametitle{Проблематика}
    \begin{columns}
        \begin{column}{0.40\textwidth}
            \includegraphics[width=1.0\linewidth]{VA.png}
        \end{column}
        \begin{column}{0.6\textwidth}
            \includegraphics[width=1.10\linewidth]{tin-copper.eps}
        \end{column}
    \end{columns}
\end{frame}

\begin{frame}
    \frametitle{Экспериментальная установка}
    \begin{figure}
        \includegraphics[width=0.6\linewidth]{detector_structure.eps}
        \caption{Устройство детектора в плоскости $x$. $\alpha$~--- угол между нормалью малого участка поверхности и электронным лучом $I_0$, $A,B,C,D$~--- детекторы, O-образец, $\Omega_{b}$~--- телесный угол детектора $D$ отклоненного на угол $\theta_b$, $H$~--- рабочее расстояние СЭМ.}
        {\label{fig:detector_structure}}% chktex 8
    \end{figure}
\end{frame}

\begin{frame}
    \frametitle{Постановка задачи}
    Согласно~\cite{PaluszynskiSlowko2005Vacuum, DrzazgaPaluszynski2005Measurement} интенсивность сигнала связана с углом наклона соответствующего участка поверхности:
    $$ I_{AB} = \frac{I_A - I_B}{I_A + I_B} \sim \tan{\alpha^x}, \quad I_{EF} = \frac{I_E - I_F}{I_E + I_F} \sim \tan{\alpha^y} $$
    \textbf{Задача 1:} Сигнал $I_{AB}(x,y), I_{EF}(x,y)$ в области $P$ предполагается известным. Необходимо восстановить градиент поверхности $\left(\frac{\partial Z}{\partial x}(x,y), \frac{\partial Z}{\partial y} (x,y) \right), (x,y) \in P$.

    \vfill

    \textbf{Задача 2:} Градиент
    \begin{equation}
        {\Big(\frac{\partial Z}{\partial x}(x,y),
        \frac{\partial Z}{\partial y} (x,y)\Big)}^{T} =
        {\big(P^x(x,y), P^y(x,y)\big)}^T, \quad (x,y) \in P
    \end{equation}
    предполагается известным. Найти функцию $Z(x,y)$ представляющую восстанавливаемую поверхность.
\end{frame}

\begin{frame}[allowframebreaks]
    \frametitle{Восстановление связи интенсивности сигнала с углом наклона участка поверхности с помощью калибровочной поверхности}


    \begin{figure}[hp]
        \includegraphics[width=0.7\linewidth]{S.eps}
        \caption{\small Разностный сигнал $I_{AB}$ в направлении сканирования вдоль оси $x$~(a) и $y$~(b).}
        {\label{fig:inputSphere}}% chktex 8
    \end{figure}

    \newpage
    Исследуемая область представляется в виде матриц $K^x$, $K^y$ размером $(N,N)$, элементами которых является интенсивность сигнала $k^x (i,j)$, $k^y (i,j)$ в точке $(x_j, y_i)$.

    Для каждого элемента $k^x_{i,j}$ матрицы $K^x$, который принадлежит калибровочной поверхности, найдем производную $\frac{\partial z}{\partial x} \big|_{(x_j,y_i)} $ аналитически, таким образом составляя таблицу, задающую сеточные значения функции обратной к аппаратной функции $\hat{F^x}$ зависимости сигнала $K^x_{i,j} = F^x (\tan(\alpha^x_{i,j}))$ от угла наклона поверхности $alpha^x$ к оси $x$.
    \begin{center}
        \begin{tabular}{| c| c |}
            \hline
            $ k^x_{i,j} $ & $ \tan{ \frac{\partial z(x_j, y_i)}{\partial x} } $ \\
            \hline
            \vdots          & \vdots \\
            \hline
            $ k^x_{i',j'} $ & $ \tan{ \frac{\partial z(x_{j'}, y_{i'})}{\partial x} } $ \\
            \hline
        \end{tabular}
    \end{center}

\end{frame}


\begin{frame}
    \frametitle{Нахождение функции, обратной к аппаратной функции микроскопа}
    \begin{figure}
        \includegraphics[width=0.9\linewidth]{Tables.eps}
        \caption{График аппаратных функций $F^x$, $F^y$ зависимости угла наклона поверхности (в градусах) от интенсивности сигнала. И их табличного представления $\hat{F^x}$, $\hat{F^y}$}
        {\label{fig:Tables}}% chktex 8
    \end{figure}
\end{frame}

\begin{frame}
    \frametitle{Восстановление градиента поверхности}
    \begin{figure}
        \includegraphics[width=0.5\linewidth]{P.eps}
        \caption{Входные изображения (a) и (b) восстанавливаемой поверхности и её карта углов (c) и (d).}
        {\label{fig:Pyr}}% chktex 8
    \end{figure}
\end{frame}

\begin{frame}[allowframebreaks]
    \frametitle{Восстановление поверхности}

    \begin{equation}
        \left\{
            \begin{split}
        &\nabla Z(x,y) = (P^x, P^y), \\
        &Z(0,0) = 0,
            \end{split}
        \right.
    \end{equation}
    \small где начальное условие выбирается произвольным образом для любой точки.

    \small Преобразуем $P^x$ следующим образом
    \begin{equation}
        P^x \equiv
        \left(
            \begin{array}{cccc}
                p_{1,1} & p_{1,2} & \cdots & p_{1,N} \\
                p_{2,1} & p_{2,2} & \cdots & p_{2,N} \\
                \vdots & \ddots & \vdots  \\
                p_{N,1} & p_{N,2} & \cdots & p_{N,N}
            \end{array}
        \right)
        \equiv{}
        \left(% chktex 37
            \begin{array}{c}
                p_{1,1} \\
                p_{2,1} \\
                \vdots \\
                p_{N,1} \\
                p_{1,2} \\
                p_{2,2} \\
                \vdots \\
                p_{1,N} \\
                \vdots \\
                p_{N,N}
            \end{array}
        \right)
        \equiv{}  B^x.
    \end{equation}

    \small Поступая аналогично с $P^y$ получим $B^y$ и с искомой функцией $Z$~---~$\hat{Z}$.
    \small Составим матрицу $A$ размером $(2N^2, N^2)$, где первые $N^2$ строк соответствуют конечно разностной аппроксимации $\frac{\partial}{\partial x}$, а последние $N^2$ строк~---~$\frac{\partial}{\partial y}$.

    \small Введем переменную сдвига индексов $s = N^2$ для второй части матрицы $A$.
    \begin{equation}
        \begin{array}{lll}
            a_{i,i} = -1/h            & for & i = 1:N^2                                       \\
            a_{i,i+N} = 1/h           & for & i = 1:N^2 - N                                   \\
            a_{i,i-N} = 1/h           & for & i = N^2 - N:N^2                                 \\
            a_{s+i+j, i+j} = -1/h     & for & i = 1, N+1, 2N+1,~\dots, (N-1)N + 1, \\
                                      & & j = 0:N-2  \\
            a_{s+i+j, i+j+1} = -1/h   & for & i = 1, N+1, 2N+1,~\dots, (N-1)N + 1, \\
                                      & & j = 0:N-2  \\
            a_{s+i+N-1, i+N-1} = -1/h & for & i = 1, N+1, 2N+1,~\dots, (N-1)N + 1             \\
            a_{s+i+N-1, i+N-2} = 1/h  & for & i = 1, N+1, 2N+1,~\dots, (N-1)N + 1
        \end{array}
    \end{equation}

    \newpage
    Составим СЛАУ
    \begin{equation}
        \label{eq:SLAE}
        A \hat{Z} = \left(
            \begin{array}{l}
                B^x \\
                B^y
            \end{array}
        \right)
        \equiv{} B
    \end{equation}

    Псевдонормальное решение которой будет являться искомая функция $Z = {( A^T A + \alpha C)}^{-1} ( A^T B) $ представляющая восстанавливаемую поверхность.
\end{frame}



\begin{frame}
    \frametitle{Результат восстановления.}
    \begin{figure}
        \includegraphics[width=0.7\linewidth]{berger.eps}
        \caption{Результат восстановления поверхности представленной на рис.~\ref{fig:Pyr}.}
        {\label{fig:berger}}% chktex 8
    \end{figure}
\end{frame}

\begin{frame}
    \frametitle{Восстановление поверхности после пробы твердости с помощью индентора Виккерса.}
    \begin{figure}
        \includegraphics[width=0.8\linewidth]{indentor-source.eps}
        \caption{Изображение $I_{AB} \; (a), I_{EF} \; (b)$ золотой пластины после теста Виккерса. Профиль синей линии представлен на рис.~\ref{fig:indentor}.}
        {\label{fig:indentor-source}}% chktex 8
    \end{figure}
\end{frame}


\begin{frame}
    \frametitle{Результат восстановления.}
    \begin{figure}
        \includegraphics[width=0.5\linewidth]{indentor.eps}
        \caption{Результат восстановления поверхности представленной на рис.~\ref{fig:indentor-source}.}
        {\label{fig:indentor}}% chktex 8
    \end{figure}
\end{frame}

\begin{frame}[allowframebreaks]
    \bibliography{bibliography}
\end{frame}

\begin{frame}
    \frametitle{Спасибо за внимание!}
    Текст презентации доступен по ссылке:
    \url{https://github.com/aborzunov/report22May}
    \begin{figure*}
        \includegraphics[width=0.5\linewidth]{repo_link.eps}
    \end{figure*}
    \usebeamerfont{institute} Contact: \url{Andrey.Borzunov@gmail.com}

\end{frame}
\end{document}
